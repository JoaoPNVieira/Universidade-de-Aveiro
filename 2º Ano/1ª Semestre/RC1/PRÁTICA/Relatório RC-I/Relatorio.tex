%%%%%%%%%%%%%%%%%%%%%%%%%%%%%%%%%%%%%%%%%%%%%%%%%%%%%%%%%
%%%%%IMPORTS:

%packages a utilizar:
\documentclass{report}
\usepackage[T1]{fontenc}     % Fontes T1
\usepackage[utf8]{inputenc}  % Input UTF8
\usepackage[top=3cm,bottom=3cm,left=3cm,right=2.5cm,asymmetric]{geometry} %fronteiras
%\usepackage[nottoc]{tocbibind}
\usepackage[table]{xcolor} %Colorir tabelas
\usepackage[backend=biber, style=ieee]{biblatex} %bibliografia
\usepackage{csquotes}  % referências
\usepackage[portuguese]{babel} %Usar língua portuguesa
\usepackage{blindtext} 
\usepackage[printonlyused]{acronym}  %Acrónimos
\usepackage{hyperref}  %Autoref no índice
\usepackage{graphicx}  %Usar imagens
\usepackage{titling}
\usepackage{multicol} %multicoluna de texto
\usepackage{adjustbox}
\renewcommand{\figurename}{Fig.} 
\renewcommand{\tablename}{Tabela} 
\usepackage[font=small,tableposition=top]{caption} 
\usepackage[font=small]{subcaption}
\usepackage{xcolor} %cor nos textos
\usepackage{amsmath} %matematica
\usepackage{amssymb} %simbolos matematicos
\graphicspath{ {./images/} } %directorio das imagens
\usepackage{fancyhdr}
\usepackage{authblk}
\usepackage{float} %Posicionamento exacto das figuras no texto
\usepackage{url}   %referencias URL
\usepackage{blindtext}
\def\UrlBreaks{\do\/\do-} %não cortar referencias

\usepackage{indentfirst} %Garantir avanço do primeiro parágrafo
\hypersetup{pdfborder=0 0 0} %Remover a caixa vermelha das referências
\usepackage{chngcntr} %Numeração contínua das figuras
\counterwithout{figure}{chapter} %Numeração contínua de figuras
\counterwithout{table}{chapter} %Numeração contínnua de tabelas
\setlength{\parskip}{0.2cm} %Aumento de espaçamento entre parágrafos

\usepackage{hyperref}

 
\begin{document}	
	%Definições do Relatório

%Dados Gerais:
\def\titulo{Projecto Final: \\ Máquina de Lavagem de Roupa}
\def\data{Junho de 2022}
\def\versao{Ver.: 1.13}
\def\departamento{Departamento de Electrónica Telecomunicações e Informática}
\def\empresa{Universidade de Aveiro}
\def\logotipo{logotipo_ua.png}

%Dados dos Autores:
%primeiro autor:
\def\pautor{João Pedro Nunes Vieira} 
\def\numpautor{Nº Mec.:  50458}
\def\contactopautor{joaopvieira@ua.pt}
%segundo autor:
\def\sautor{Leandro Roque Costa} 
\def\numsautor{Nº Mec.: 110326}
\def\contactosautor{lrc@ua.pt}
%
\def\autores{\pautor \\ \sautor}

	%Capa do Relatório:
\begin{titlepage} 
	\begin{center}
	\includegraphics[scale=0.50]{\logotipo}
	\line(1,0){350} \\ 
		\vspace*{2mm}
	{\Large \uc} \\
		\vspace*{2mm} 	
	{\Huge \titulo} \\
		\vspace*{2mm}
	\line(1,0){350} \\ 
		\vspace*{2mm}
	{\Large \empresa} \\
		\vspace*{20mm}
	{\Large \autores} \\ 
		\vspace*{\fill}
	\end{center}

	\begin{flushright} 
		{\large \departamento} \\ 
		{\versao} 
	\end{flushright}

\end{titlepage}

	%Página de Título:
\predate{\begin{flushright}\small}
\postdate{\par\end{flushright}}

\title{ 
	{\huge\textbf{\titulo} } \\ 
	{\large \departamento\\ \empresa} }

\author{

    \begin{tabular}{l}
        \pautor, \numpautor\hfil \\
        \contactopautor
    \end{tabular}
    \and
    \begin{tabular}{l}
        \sautor, \numsautor\hfil \\
        \contactosautor
    \end{tabular}
    \and
    \begin{tabular}{l}
        \tautor, \numtautor\hfil \\
        \contactotautor
    \end{tabular}
    \and
    \begin{tabular}{l}
        \qautor, \numqautor\hfil \\
        \contactoqautor
    \end{tabular}
}




\date{\vspace{\fill}{\data}} 
\maketitle
	\begin{abstract}

O presente relatório aborda a comunicação entre aplicações seguindo um modelo Cliente-Servidor, usando conceitos abordados no decorrer da Unidade Curricular de Laboratórios de Informática, nomeadamente programação de sockets TCP, Criptografia, Documentação JSON, programação em Python, etc... \\
Este trabalho é relevante pois com o avanço tecnológico actual, a aprendizagem dos conceitos anteriormente referidos torna-se imperativa para os estudantes de cursos ligados à tecnologia de informação.\\
O projecto foi realizado por duas pessoas, tendo como objectivo o desenvolvimento das capacidades de trabalho em grupo, desenvolvimento e estruturação de código, realização de testes funcionais e planeamento através da plataforma \textbf{code.ua.pt}. \\
Foi possível implementar com sucesso um modelo de comunicação entre aplicações Cliente-Servidor em gênero "Jogo High-Low" com funcionalidade de segurança(encriptação) de forma simples, prática e eficaz, conseguindo-se corrigir todos os erros encontrados.

\end{abstract}
	
%%%%%%%%%%%%%%%%%%%%%%%%%%%%%%%%%%%%%%%%%%%%%%%%%%%%%%%%%
%%%%%INDICE:

\renewcommand{\contentsname}{Índice}
\tableofcontents
\listoffigures
\pagenumbering{roman}

%%%%%%%%%%%%%%%%%%%%%%%%%%%%%%%%%%%%%%%%%%%%%%%%%%%%%%%%%
%%%%%ACRÓNIMOS:

\chapter*{Acrónimos}
\begin{acronym}

\acro{ua}[UA]{Universidade de Aveiro}
\acro{leci}[LECI]{Licenciatura em Engenharia de Computadores e Informática}
\acro{uc}[UC]{Unidade Curricular}
\acro{rci}[RC-I]{Redes de Comunicações I}
\acro{ipv4}[IPv4]{Internet Protocol version 4}
\acro{ipv6}[IPv6]{Internet Protocol version 6}
\end{acronym}

%%%%%%%%%%%%%%%%%%%%%%%%%%%%%%%%%%%%%%%%%%%%%%%%%%%%%%%%%
%%%%%HEADERS & FOOTERS:

\pagestyle{fancy}
\fancyhf{}
\rhead{\titulo}
\lhead{Introdução}
\cfoot{\thepage}

%%%%%%%%%%%%%%%%%%%%%%%%%%%%%%%%%%%%%%%%%%%%%%%%%%%%%%%%%
%%%%%INTRODUÇÃO:

\chapter{Introdução Teórica}
\label{chap.Intro}
\pagenumbering{arabic}
\begin{multicols}{2}

Introdução teória (...)
 
\end{multicols}

\section{Software}
\label{sec.software}



\begin{itemize}
	\item \textsl{item:} (...)
\end{itemize}

\clearpage

%%%%%%%%%%%%%%%%%%%%%%%%%%%%%%%%%%%%%%%%%%%%%%%%%%%%%%%%%
%%%%%1ª PARTE: Implementação

\chapter{Implementação}	
\label{chap.implementacao}
\lhead{Implementação}

(...)

%%%%%%%%%%%%%%%%%%%%%%%%%%%%%%%%%%%%%%%%%%%%%%%%%%%%%%%%%
%%%%%2ª Parte: Testes
%
\chapter{Análise de Resultados:}
\label{chap.analise}
\lhead{Conclusão}

(...)

%
%%%%%%%%%%%%%%%%%%%%%%%%%%%%%%%%%%%%%%%%%%%%%%%%%%%%%%%%%
%%%%%CAPÍTULO 3: CONCLUSÃO
%
\chapter{Conclusão:}
\label{chap.conclusão}
\lhead{Conclusão}

(...)

%
%%%%%%%%%%%%%%%%%%%%%%%%%%%%%%%%%%%%%%%%%%%%%%%%%%%%%%%%%
%%%%%BIBLIOGRAFIA:
%
\begin{thebibliography}{9}

\bibitem{thinkpython} 
Allen Downey
\textit{Think Python - How to Think Like a Computer Scientist}. |
Green Tea Press, 2nd Edition, Version 2.4.0, 2015

\bibitem{pycryptome}
PyCryptodome documentation. |
\text{ pycryptodome.readthedocs.io, acedido 18/05/2021.}

\bibitem{Wikipedia}
Wikipédia: Enciclopédia livre. |
\text{ pt.wikipedia.org, acedido 19/05/2021.}

\bibitem{serverfault}
ServerFault Website. |
\text{ serverfault.com, acedido 22/05/2021.}

\bibitem{iana}
Internet Assigned Numbers Authority. |
\text{ www.iana.org, acedido 22/05/2021.}


\end{thebibliography}
\end{document}
