\begin{abstract}

O presente relatório aborda o desenvolvimento de um website do estilo loja online, onde foram implementadas duas soluções distintas relativamente à segurança informática, cujo o objectivo é a venda merchandise do Departamento de Electrónica, Telecomunicações e Informática da Universidade de Aveiro usando conceitos abordados no decorrer da Unidade Curricular de Segurança Informática e Nas Organizações.
\paragraph{}
Este trabalho é relevante pois permite a aplicação de conceitos adquiridos no contexto de aula teórica, nomeadamente o foco na previsibilidade de sistemas, processos, ambientes etc... de soluções informáticas seguras, tais como o planeamento, desenvolvimento, execução, análise e auditoria, com especial foco na defesa contra actividades não autorizadas fazendo prevalecer o funcionamento previsto da solução recorrendo a políticas de segurança pré estabelecidas, os "Objectivos da Segurança da Informação" \acf{cia} e conhecimento sobre vulnerabilidades comuns \acf{cwe} a serem evitadas e/ou corrigidas.
\paragraph{}
O projecto foi realizado por quatro pessoas, tendo como objectivos a consolidação de conceitos adquiridos, desenvolvimento de capacidades de deteção e prevenção de \ac{cwe} e a realização de análise e auditoria de falhas, erros, comportamentos desviantes, exposições presentes e potenciais impactos da exploração maliciosa destas. 
\paragraph{}
Foi possível implementar com sucesso duas versões do website referido: uma vulnerável com a aplicação de \ac{cwe} e uma corrigida com funcionalidades de segurança de forma. Todo o presente trabalho realizado encontra-se disponível em \textbf{\url{https://github.com/detiuaveiro/1st-project-group_26}}.

\end{abstract}