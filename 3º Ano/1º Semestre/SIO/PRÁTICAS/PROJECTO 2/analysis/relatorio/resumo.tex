\begin{abstract}

Este relatório aborda a auditoria e implementação de correções de software para uma loja online desenvolvida como parte do Projeto Nº1 \textbf{( \url{https://github.com/detiuaveiro/1st-project-group_26}  )} da Unidade Curricular de \ac{sio}, no decorrer do ano letivo 2023/24. \\
O website é uma loja online destinada à venda de merchandise do  \ac{deti} da \ac{ua}, cujo objetivo principal é melhorar a funcionalidade e a segurança da loja original, seguindo os procedimentos definidos no \textbf{ \ac{asvs}} Nível 1.

A abordagem do projeto inclui:

\begin{enumerate}
    \item \textbf{Auditoria do Website:} Realização de uma auditoria de conformidade com os requisitos do Nível 1 do \ac{asvs}.
    \item \textbf{Implementação de Melhorias de Segurança:} Abordagem das questões identificadas durante a auditoria para cumprir os requisitos do Nível 1 do \ac{asvs}.
    \item \textbf{Documentação Detalhada:} Elaboração de documentação detalhada sobre o impacto dos problemas identificados e das correções implementadas.
    \item \textbf{Novas Funcionalidades Implementadas:}
        \begin{itemize}
            \item \textbf{Password strength evaluation:} exigindo um mínimo de força para passwords, de acordo com V2.1 e com verificação de violação recorrendo a um serviço externo.
            \item \textbf{Autenticação de Dois Fatores (MFA):} utilizando Google Identity baseado em OAuth 2.0 com \ac{oidc}.
        \end{itemize}
\end{enumerate}

\paragraph{}

Este projeto alcançou com sucesso os objetivos propostos, destacando-se a realização de uma auditoria abrangente, a implementação eficaz das melhorias de segurança identificadas e a introdução bem-sucedida de novas funcionalidades. O repositório \url{https://github.com/detiuaveiro/2nd-project-group_26} contém todos os elementos necessários para uma compreensão completa do processo.

\end{abstract}