
\begin{abstract}

O presente relatório aborda o desenvolvimento de um sistema que permita a criação e reprodução de músicas e sons fazendo recurso ficheiros de audio, usando conceitos e temas abordados no decorrer da Unidade Curricular de Laboratórios de Informática da Universidade de Aveiro, nomeadamente servidores aplicacionais web, bases de dados, transformação e transmissão de informação sonora e aplicações web para serviços móveis  (android/ios...). A aprendizagem destes conceitos anteriormente referidos torna-se imperativa para os estudantes de cursos superiores ligados às tecnologias de informação, já que é evidente o papel fundamental que estas desempenham no desenvolvimento moderno de tecnológias web. \\
O projecto foi realizado por uma equipa de quatro pessoas, tendo como objectivo o desenvolvimento das capacidades de trabalho em grupo, desenvolvimento e estruturação de código, realização de testes funcionais e planeamento através da plataforma \textbf{code.ua.pt}. \\
Foi possível desenvolver a implementação uma interface web para comunicação entre o utilizador e os ficheiros servidos para interação, uma aplicação web (API) para servir e comunicar diversos ficheiros e informações entre todos os elementos programáticos, uma base de dados desenvolvida em SQL para armazenar informação sobre diversos ficheiros de audio, respectivos autores, sistemas de votação por utilizadores e outras informações relevantes e a implementação de um gerador/compositor virtual de música. Foi possivel corrigir todos os erros encontrados através de testes executados por todos os elementos e o projecto foi terminado pela equipa de forma satisfatória. \\
Serve ainda o presente relatório como base de documentação para o projecto.


\end{abstract}